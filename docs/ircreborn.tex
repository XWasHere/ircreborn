\input texinfo
@node Top, invoking ircreborn, (dir), (dir)
welcome to the ircreborn manual! sorry that it's a bit lacking at the moment

@menu
* invoking ircreborn::
* ircreborn client::
* ircreborn server::
@end menu
@node invoking ircreborn, ircreborn client, Top, Top
ircreborn only has a few arguments.

the client can be started with @option{--client}

the server can be started with @option{--server}
@node ircreborn client, invoking the client, invoking ircreborn, Top
heres the docs for the client

@menu
* invoking the client::
* configuring the client::
@end menu
@node invoking the client, configuring the client, ircreborn client, ircreborn client
the client is just started with @w{"ircreborn ---client"}, it currently has no arguments
@node configuring the client, server config block, invoking the client, ircreborn client
the ircreborn client is configured in "~/.ircreborn/client"
the format it uses is explained in the following sections

@menu
* server config block::
@end menu
@node server config block, ircreborn server, configuring the client, configuring the client
this is the server config block :)

@example
server "<SERVER NAME>" @{
    <OPTIONS>
@}
@end example

options can be:

@example
host "<IPV4 ADDRESS>"
@end example

@example
port <PORT>
@end example

@example
nick "<NAME>"
@end example

heres an example:

@example
server "example server" @{
    host "127.0.0.1"
    port 10010
    nick "xwashere"
@}
@end example

@node ircreborn server, invoking the server, server config block, Top

@menu
* invoking the server::
* configuring the server::
@end menu
@node invoking the server, configuring the server, ircreborn server, ircreborn server
the ircreborn server is started using @w{"ircreborn ---server"}

there is also one argument: @option{--listen-port <PORT>} which will override the listen port in the server config
@node configuring the server,, invoking the server, ircreborn server
the server config file follows a very similar format to the client config. heres how it works

heres the listen option, it specifies what port the server will listen on. (it defaults to 10010)

@example
listen <PORT>
@end example

heres an example config

@example
listen 10010
@end example